\documentclass[journal,12pt,twocolumn]{IEEEtran}
\usepackage{amsthm}
\usepackage{gensymb}
\usepackage{setspace}
\singlespacing
\usepackage[cmex10]{amsmath}
\usepackage{bm}

\usepackage{cases}
\usepackage{mathrsfs}
\usepackage{cite}
\usepackage{stfloats}
\usepackage{mathtools}
\usepackage[breaklinks=true]{hyperref}
\usepackage{graphicx}
\usepackage{subfig}
\usepackage{txfonts}
\usepackage{longtable}
\usepackage{multirow}
\usepackage{tfrupee}
\usepackage{enumitem}
\usepackage{tikz}
\usepackage{steinmetz}
\usepackage{verbatim}
\usepackage{circuitikz}
\usepackage{tkz-euclide}





\usetikzlibrary{calc,math}
\usepackage{listings}
    \usepackage{color}                                            %%
    \usepackage{array}                                            %%
    \usepackage{longtable}                                        %%
    \usepackage{calc}                                             %%
    \usepackage{multirow}                                         %%
    \usepackage{hhline}                                           %%
    \usepackage{ifthen}                                           %%
    \usepackage{lscape}     
\usepackage{multicol}
\usepackage{chngcntr}

\DeclareMathOperator*{\Res}{Res}

\renewcommand\thesection{\arabic{section}}
\renewcommand\thesubsection{\thesection.\arabic{subsection}}
\renewcommand\thesubsubsection{\thesubsection.\arabic{subsubsection}}

\renewcommand \thesectiondis{\arabic{section}}
\renewcommand\thesubsectiondis{\thesectiondis.\arabic{subsection}}
\renewcommand\thesubsubsectiondis{\thesubsectiondis.\arabic{subsubsection}}


\hyphenation{op-tical net-works semi-conduc-tor}
\def\inputGnumericTable{}                                 %%

\lstset{
%language=C,
frame=single, 
breaklines=true,
columns=fullflexible
}
\begin{document}

\newcommand{\BEQA}{\begin{eqnarray}}
\newcommand{\EEQA}{\end{eqnarray}}
\newcommand{\define}{\stackrel{\triangle}{=}}
\bibliographystyle{IEEEtran}
\raggedbottom
\setlength{\parindent}{0pt}
\providecommand{\mbf}{\mathbf}
\providecommand{\pr}[1]{\ensuremath{\Pr\left(#1\right)}}
\providecommand{\qfunc}[1]{\ensuremath{Q\left(#1\right)}}
\providecommand{\sbrak}[1]{\ensuremath{{}\left[#1\right]}}
\providecommand{\lsbrak}[1]{\ensuremath{{}\left[#1\right.}}
\providecommand{\rsbrak}[1]{\ensuremath{{}\left.#1\right]}}
\providecommand{\brak}[1]{\ensuremath{\left(#1\right)}}
\providecommand{\lbrak}[1]{\ensuremath{\left(#1\right.}}
\providecommand{\rbrak}[1]{\ensuremath{\left.#1\right)}}
\providecommand{\cbrak}[1]{\ensuremath{\left\{#1\right\}}}
\providecommand{\lcbrak}[1]{\ensuremath{\left\{#1\right.}}
\providecommand{\rcbrak}[1]{\ensuremath{\left.#1\right\}}}
\theoremstyle{remark}
\newtheorem{rem}{Remark}
\newcommand{\sgn}{\mathop{\mathrm{sgn}}}
\providecommand{\abs}[1]{\vert#1\vert}
\providecommand{\res}[1]{\Res\displaylimits_{#1}} 
\providecommand{\norm}[1]{\lVert#1\rVert}
%\providecommand{\norm}[1]{\lVert#1\rVert}
\providecommand{\mtx}[1]{\mathbf{#1}}
\providecommand{\mean}[1]{E[ #1 ]}
\providecommand{\fourier}{\overset{\mathcal{F}}{ \rightleftharpoons}}
%\providecommand{\hilbert}{\overset{\mathcal{H}}{ \rightleftharpoons}}
\providecommand{\system}{\overset{\mathcal{H}}{ \longleftrightarrow}}
	%\newcommand{\solution}[2]{\textbf{Solution:}{#1}}
\newcommand{\solution}{\noindent \textbf{Solution: }}
\newcommand{\cosec}{\,\text{cosec}\,}
\providecommand{\dec}[2]{\ensuremath{\overset{#1}{\underset{#2}{\gtrless}}}}
\newcommand{\myvec}[1]{\ensuremath{\begin{pmatrix}#1\end{pmatrix}}}
\newcommand{\mydet}[1]{\ensuremath{\begin{vmatrix}#1\end{vmatrix}}}
\numberwithin{equation}{subsection}
\makeatletter
\@addtoreset{figure}{problem}
\makeatother
\let\StandardTheFigure\thefigure
\let\vec\mathbf
\renewcommand{\thefigure}{\theproblem}
\def\putbox#1#2#3{\makebox[0in][l]{\makebox[#1][l]{}\raisebox{\baselineskip}[0in][0in]{\raisebox{#2}[0in][0in]{#3}}}}
     \def\rightbox#1{\makebox[0in][r]{#1}}
     \def\centbox#1{\makebox[0in]{#1}}
     \def\topbox#1{\raisebox{-\baselineskip}[0in][0in]{#1}}
     \def\midbox#1{\raisebox{-0.5\baselineskip}[0in][0in]{#1}}
\vspace{3cm}
\title{Assignment3}%number
\author{CS20Btech11035 -NYALAPOGULA MANASWINI}
\maketitle
\newpage
\bigskip

\renewcommand{\thefigure}{\theenumi}
\renewcommand{\thetable}{\theenumi}
Download python code from 
\begin{lstlisting}
https://github.com/N-Manaswini23/assignment3/blob/main/assignment3.py
\end{lstlisting}
%
Download latex code from 
\begin{lstlisting}
https://github.com/N-Manaswini23/assignment3/blob/main/assignment3.tex
\end{lstlisting}
%

\section*{GATE XE-C QUESTION 17}
Box-S has $2$ white and $4$ black balls and box-T has $5$ white and $3$ black balls.A ball is drawn at random from box-S and put in box-T.Subsequently,the probability of drawing a white ball from box-T is? (rounding off to $ 2 $decimal places)

\section*{SOLUTION}
Box-S has $2$ white and $4$ black balls.\\
Box-T has $5$ white and $3$ black balls.\\
\begin{table}[h!]
\resizebox{9.5cm}{!}
{ 
\begin{tabular}{|c|c|}
\hline
Event & definition \\
\hline
A & Event of transfering white ball from\\
& box-S to box-T\\
\hline
B & Event of transfering black ball from\\
& box-S to box-T\\
\hline
C  & Event of drawing white ball from box-T \\
\hline
$\pr{C|A}$ & Probability of drawing whiteball from \\
& box-T after transfering white ball to box-T.\\
\hline
$\pr{C|B}$ & Probability of drawing whiteball from\\
& box-T after transfering black ball to box-T.\\
\hline
\end{tabular}
}
\caption{Table 1} 
\label{tab:1}
\end{table}


\begin{table}[h!]
\resizebox{9.5cm}{!}
{ 
\begin{tabular}{|c|c|c|c|c|}
\hline
Probability & $\pr{A}$ & $\pr{B}$ & $\pr{C|A}$ & $\pr{C|B}$ \\
\hline
value & $\frac{1}{3}$ &  $\frac{2}{3}$ &  $\frac{6}{9}$ &  $\frac{5}{9}$ \\
\hline
\end{tabular}
}
\caption{Table 2} 
\label{tab:2}
\end{table}


From Baye's theorem
\begin{align}
\pr{\text{drawn ball is white}}&= \pr{C}\\
&=\pr{C|A} \times \pr{A} \notag \\
 & +\pr{C|B} \times \pr{B}  \label{5}
\end{align}
Substiting values from table \eqref{tab:2} in \eqref{5}
\begin{align}
\pr{C} &= \frac{6}{9} \times \frac{1}{3}  + \frac{5}{9} \times \frac{2}{3} \\
&=\frac{16}{27} \label{6}
\end{align}
$\therefore$ Probability of drawing white ball from\\
 box-T $=\pr{C}=\frac{16}{27}=0.59$
\begin{figure}[htb!]
\begin{center}
\includegraphics[width=0.5\textwidth]{assignment3_drawn_theoretrical.png}
\end{center}
\end{figure}

\begin{figure}[htb!]
\begin{center}
\includegraphics[width=0.5\textwidth]{assignment3_drawn_sim.png}
\end{center}
\end{figure}
\end{document}
